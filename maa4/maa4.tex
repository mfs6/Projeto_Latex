\documentclass{article}
\usepackage[utf8]{inputenc}

\title{IF796 - Mineração na Web}
\author{Matheus Alves Almeida }
\date{Maio 2019}

\usepackage{natbib}
\usepackage{graphicx}

\begin{document}

\maketitle

\section{Introdução}
Um grande problema enfrentado hoje em dia é ensinar as pessoas a ignorarem a informação irrelevante que vem em excesso no cotidiano. O que é chamado de "Information Overload"\citep{r3}. 
A Mineração na Web surge com essa motivação, de evitar que as pessoas "morram ignorantes em um mar de informações" e diminuir a dificuldade de encontrar documentos relevantes\citep{r2}. Ela funciona a partir de uma consulta do usuário, no qual são pegos os itens de informação ou palavras-chave e retornam um conjunto de documentos ordenados que são relevantes para aquela consulta realizada pelo usuário\citep{r4}. Além disso, a mineração tem um papel importante na competitividade dos negócios atualmente. Um sistema de negócios parte de uma tentativa de fornecimento pleno de informações, partindo dos dados disponíveis geradores de informação construindo indicadores e realizando a identificação de tendências que apoiem a tomada de decisão.   

\begin{figure}[h!]
\centering
\includegraphics[scale=0.5]{figura1}
\caption{Funcionamento básico da mineração na web\citep{r1}}
\label{fig:figura1}
\end{figure}

\section{Relevância}
São vários os pontos relevantes dessa disciplina para um cientista da computação. Os conceitos aprendidos tem diversas aplicações no mundo real, que não se resumem a apenas engenhos de buscas (Google, Yahoo), mas também abrangem sistemas de recomendação e de extração de informação, bem como agentes notificadores e chatbots. Estas são algumas das grandes aplicações da mineração que só tende a crescer e se aperfeiçoar.       

\section{Relação com outras disciplinas}
\begin{table}[]
\begin{tabular}{|c|l|lll}
\cline{1-2}
Disciplina                                                                             & \multicolumn{1}{c|}{Relação}                                                                                                                                                                         &  &  &  \\ \cline{1-2}
\begin{tabular}[c]{@{}c@{}}IF685 - Gerenciamento de Dados \\ e Informação\end{tabular} & \begin{tabular}[c]{@{}l@{}}As informações de interesse do usuário \\ devem ser armazenadas em Bancos de \\ Dados, para futura recuperação.\end{tabular}                                              &  &  &  \\ \cline{1-2}
IF684 - Sistemas Inteligentes                                                          & \begin{tabular}[c]{@{}l@{}}Trata de sistemas de recomendação de \\ itens, análise de opiniões (classificação\\ de opiniões entre positivas e negativas),\\ e outros assuntos avançados.\end{tabular} &  &  &  \\ \cline{1-2}
\begin{tabular}[c]{@{}c@{}}IF672 - Algoritmos e Estruturas\\  de Dados\end{tabular}    & \begin{tabular}[c]{@{}l@{}}Uso de algoritmos para ordenar os \\ dados e optimizar a execução.\end{tabular}                                                                                           &  &  &  \\ \cline{1-2}
\end{tabular}
\end{table}
     

\bibliographystyle{plain}
\bibliography{references}
\end{document}
