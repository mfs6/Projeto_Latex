\documentclass[10pt]{article}
\usepackage[utf8]{inputenc}
\usepackage[portuguese]{babel}
\usepackage{natbib}
\usepackage{graphicx}

\title{IF678 - Infra-estrutura de Comunicação}
\author{Michael Felipe dos Santos }
\date{Maio 2019}

\begin{document}

\maketitle

\section{Introdução}
A disciplina Infra-estrutura de Comunicação (IF678), apresenta diversos conceitos relacionados às formas de comunicação, com foco nas redes de computadores, tais como: protocolos de rede,  meios físicos de comunicação, estrutura da internet, camadas de protocolo e modelos de serviço, segurança de redes, história da Internet.
\cite{sitedadisciplina}

Esses e outros tópicos são trabalhados de forma gradativa durante o semestre, visando uma melhor compreensão por parte do aluno a respeito das várias tecnologias de comunicação, usando como exemplo a internet.

\begin{figure}[h!]
    \centering
    \includegraphics[width=5cm,heigth=5cm]{internet}
    \caption{Representação de uma rede \cite{figura1}}
    \label{fig:internet}
\end{figure}

\section{Relevância}
Esta disciplina é de suma importância no contexto do curso de Ciência da Computação, pois introduz conceitos a respeito do funcionamento das infra-estruturas de comunicação.

A disciplina tem como pontos positivos, a explicação de diversos conceitos que ajudam o aluno a se tornar um excelente profissional na área de redes de computadores, ao passo que estuda, com detalhes, conceitos como os meios físicos de comunicação, redes de acesso sem fio, camadas de protocolos, os diversos protocolos de rede e etc. \cite{slide} Todo este conhecimento, com certeza, faz o aluno largar na frente no mercado de trabalho, nessa área.

Porém, como ponto negativo, é válido citar que, o conteúdo é bastante extenso e muito técnico\cite{gradecurricular}, o que torna a disciplina um pouco cansativa, algo que pode representar uma certa dificuldade por parte de alguns discentes.

\section{Relação}

Na grade curricular do curso encontram-se disciplinas que se relacionam diretamente com a cadeira IF678 - Infra-estrutura de Comunicação, tais como:

\begin{table}[h!]
\begin{tabular}{cclll}
\cline{1-2}
\multicolumn{1}{|c|}{\textbf{Cadeira}} & \multicolumn{1}{c|}{\textbf{Relação}} &  &  &  \\ \cline{1-2}
\multicolumn{1}{|c|}{IF738 - Redes de Computadores} & \multicolumn{1}{c|}{\begin{tabular}[c]{@{}c@{}}A disciplina aborda conceitos como as\\  camadas de protocolos e o modelo de\\  referência OSI, tendo por objetivo\\  fornecer ao aluno uma visão abrangente\\  da área de redes de computadores. \cite{IF738}\end{tabular}} &  &  &  \\ \cline{1-2}
\multicolumn{1}{|c|}{IF741 - Gerenciamento de Redes} & \multicolumn{1}{c|}{\begin{tabular}[c]{@{}c@{}}Apresenta conceitos relacionados ao\\  gerenciamento do funcionamento das\\  redes de computadores, com a utilização\\  de técnicas de monitoramento, usando \\ por exemplo, protocolos como o SNMP. \cite{IF741}\end{tabular}} &  &  &  \\ \cline{1-2}
\multicolumn{1}{l}{} & \multicolumn{1}{l}{} &  &  & 
\end{tabular}
\end{table}

\bibliographystyle{plain}
\bibliography{mfs6}
\end{document}
