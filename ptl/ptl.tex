\documentclass{article}

\usepackage[brazil]{babel}
\usepackage[utf8x]{inputenc}
\usepackage[T1]{fontenc}
\usepackage{array}

\usepackage[a4paper,top=3cm,bottom=2cm,left=3cm,right=3cm,marginparwidth=1.75cm]{geometry}

\usepackage{amsmath}
\usepackage{graphicx}
\usepackage[colorinlistoftodos]{todonotes}
\usepackage[colorlinks=true, allcolors=blue]{hyperref}


\title{IF690 - História e Futuro da Computação}
\author{Pedro Tenório Lemos}
\date{May 2019}

\begin{document}
\maketitle

\section{Introdução}

História e Futuro da Computação ou, popularmente, HFC é uma disciplina do curso de Ciência da Computação que lida com a evolução das tecnologias ligadas à computação com o passar do tempo, bem como as expectativas para os futuros paradigmas dessa área. \cite{cinwikiif690} Essa disciplina se envolve com praticamente todos os ramos da computação, pois todos passaram por evoluções desde sua criação até os dias atuais. Ela trabalha assuntos como: Evolução de Hardware/Software, Computadores quânticos/biológicos, Novas linguagens e ferramentas. \cite{cingradif690}

\begin{figure}[h!]
    \centering
    \includegraphics[scale=0.5]{Evolution.jpg}
    \caption{Evolução do Computador \cite{figura}}
    \label{fig:evolution}
\end{figure}

\section{Relevância}
Basicamente, a relevância desta disciplina está atrelada à relevância da História em si, tendo o mesmo papel de investigação e busca pela evolução. O conhecimento sobre o passado e os erros cometidos nele nos faz acertar mais no presente, assim como os acertos nos fazem ter noção de que, às vezes, o impossível está ao nosso alcance. \cite{hdccleuzio} Outra abordagem interessante do curso é a discussão da História da Ciência e como ela veio a se tornar a principal forma de obtenção de conhecimento a partir da Idade Moderna. Além disso, o curso estimula o pensamento científico e tenta dismistificar algumas ideias "autoritárias" da Ciência moderna como a impossibilidade de multiplas explicações para o mesmo fenômeno. \cite{ssrthomas}  

\subsection{Pontos Positivos}

\begin{itemize}
    \item Desenvolve o pensamento científico e cria as bases para o mundo acadêmico.
    \item Discute temas importantes para o desenvolvimento da computação e do mundo contemporâneo.
    \item Apresenta tecnologias novas e em desenvolvimento, pavimentando o caminho para o novo mercado.
\end{itemize}

\subsection{Pontos Negativos}

\begin{itemize}
    \item É, por alguns, considerada uma disciplina chata e que não apresenta conhecimentos úteis para o resto do curso.
    \item Consome bastante tempo em um período avançado do curso.
    
\end{itemize}

\section{Relação com Outras Disciplinas}

HFC se relaciona com praticamente todas as disciplinas do curso, pois estuda o processo que levou às duas criações, mas algumas delas merecem atenção especial:

\begin{table}[h]
    \centering
    \begin{tabular}{||c|m{8.5cm}||}
    \hline
     Disciplina & \\
     \hline
     IF668 - Introdução a Computação & Essa disciplina aborda, em seu projeto de ensino, uma parte da História da Computação para alunos de primeiro período. \cite{cinwikiif668}\\
     \hline
     IF679 - Informática e Sociedade & Essa disciplina aborda os impactos sociais da informática, outro ponto importante de se estudar História. \cite{cinwikiif679}\\
     \hline
     
     
        
    \end{tabular}
    
    
    \label{tab:relacoes}
\end{table}

\bibliographystyle{alpha}
\bibliography{ptl}
\nocite{*}


\end{document}
