\documentclass{article}
\usepackage[utf8]{inputenc}
\usepackage{array}
\usepackage{graphicx}
\usepackage{natbib}
\usepackage[brazil]{babel}
\usepackage[T1]{fontenc}
\begin{document}
\title{IF674 - Infra-Estrutura de Hardware}
\maketitle
\author{Nalbert Carvalho de Melo}
\section{Introdução}
O curso de Infra-Estrutura de Hardware tem por objetivo dar uma visão geral sobre os componentes dos computadores, que são:  processador, sistema de memória, entrada e saída e barramentos. Quando se tratando de processadores um tema muito abordado será CPU, conceitos básicos e o que carecteriza a mesma, mas não só conceitos básicos serão estudados, pipeline e super-escalares estão no programa do curso. Os computadores modernos possuem memórias com diferentes características e parte do curso trata de entender o funcionamento de cada uma dessas memórias e suas aplicações\citep{ref1}. 
\begin{figure}[h!]
    \centering
    \includegraphics[scale=0.5]{intro.png}
    \caption{Imagem que retrata alguns dos temas tratados em Infra-Estrutura de Hardware e como eles se relacionam entre eles.}
    \citep{ref5}
    
    \label{fig:imagem}
\end{figure}

\section{Relevância}
A computação moderna necessita profissionais fluentes tanto em software quanto em hardware, com base nisso, saber sobre ambas as áreas possibilita um entedimento mais profundo sobre os fundamentos da computação. Seja o interesse em hardware ou software, o obejtvo do curso de Infra-Estrutura de Hardware é mostrar como ambos se completam. Enquanto existem os programadores que estudam ambas as áreas existem aqueles que não estudam, esses dependem de outros profissionais da computação como engenheiros e arquitetos da computação para que seus programs rodem mais rápido, logo os profissionais que possuem conhecimento tanto sobre software quanto hardware saem na frente do mercado de trabalho\citep{ref2}.

\section{Relação com outras disciplinas}
\begin{center}
\begin{tabular}{ |c|m{25em}| } 
 \hline
Infra-Estrutura de Software & Esse é um dos cursos que faz parte da tríade comunicação, software e hardware.Esta apresenta conceitos básicos de software, que como antes dito se relacionam profundamente com o estudo de hardware.\citep{ref3} \\ 
\hline
 Sistemas  Digitais  &  Este curso visa dar ao aluno conhecimentos de circuitos lógicos digitais combinacionais e sequênciais, tratando assim temas próximos aos do curso de Infra-Estrutura de Hardware.\citep{ref3} \\ 
\hline
Infra-Estrutura de Comunicação & Este curso trata alguns aspectos físicos do computador além de como  supracitado, esse curso também faz parte da tríade, e é complementar ao curso de Infra-Estrutura de Software que por tabela se relaciona como o curso de Hardware. \citep{ref4}\\ 
 \hline
\end{tabular}
\end{center}


\bibliographystyle{plain}
\bibliography{bibly.bib}

\end{document}