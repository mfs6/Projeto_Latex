\documentclass[10pt]{article}
\usepackage[utf8]{inputenc}
\usepackage{float}
\usepackage[portuguese]{babel}
\usepackage{graphicx}
\date{\vspace{-5ex}}


\title{IF673 - Lógica para Computação}
\author{Clara Ribeiro de Santana }


\usepackage{natbib}

\begin{document}

\maketitle

\section{Introdução}
A disciplina de Lógica para Computação, tem como principal objetivo promover o entendimento do aluno sobre a ciência do raciocínio, inferência e dedução. "A Lógica Matemática estuda as noções de validade e consistência de argumentos utilizando elementos da Matemática, tais como a teoria dos conjuntos e a álgebra booleana." \citep{log}

O estudo da disciplina de Lógica para computação envolve o método formal-dedutivo de raciocínio, noções de prova e refutação e representações simbólica.

\begin{figure}[h!]
\centering
\includegraphics[scale=0.3]{imagemgit.jpg}
\caption{Livro de Lógica utilizado na disciplina}
\citep{livro1}
\label{fig:imagemgit}
\end{figure} 

\section{Relevância}
Lógica para Computação tem envolvimento direto com a noção de softwares de computação. Posto que, algo programável também significa que teve uma dedução formal e previamente pensada para ter uma melhor funcionalidade, independente da linguagem utilizada.

\newpage

Pontos positivos

\begin{itemize}
   \item Aprender a pensar nos problemas de programação de forma lógica e eficiente.
 \end{itemize}
Pontos negativos
\begin{itemize}
   \item Acredito que esta disciplina não apresente pontos negativos.
 \end{itemize}

\section{Relação com outras disciplinas}



\begin{table}[h]
 \centering
 {\renewcommand\arraystretch{1.25}
 \caption{Comparação}
 \begin{tabular}{ l l }
  \cline{1-1}\cline{2-2}  
   
  \\  
  \cline{1-1}\cline{2-2}  
    \multicolumn{1}{|p{3.850cm}|}{IF670 - Matemática Discreta} &
    \multicolumn{1}{p{4.217cm}|}{Matemática Discreta é um pré-requesito para cursar a disciplina de Lógica para Computação, nela aprendemos conceitos básicos que serão utilizados em Lógica.  \citep{matematica}}
  \\  
  \cline{1-1}\cline{2-2}  
    \multicolumn{1}{|p{3.850cm}|}{IF682 - Engenharia Software e Sistemas } &
    \multicolumn{1}{p{4.217cm}|}{Como o foco desta disciplina é programar linguagens e aprender processos que ajudam na construção do software (como o planejamento de execução, que vai se utilizar o que foi aprendido na disciplina de Lógica). \citep{eng}}
  \\  
  \hline

 \end{tabular} }
\end{table}

\bibliographystyle{plain}
\bibliography{crs5}
\end{document}
